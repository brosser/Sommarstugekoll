\documentclass[a4paper,11pt]{article}
\usepackage[T1]{fontenc}
\usepackage[utf8]{inputenc}
\usepackage{lmodern}
\usepackage{hyperref}
\usepackage{graphicx}
\usepackage{rotating}
\usepackage{listings}
\usepackage{color}

% Swedish
% \usepackage[swedish]{babel}

% Table of contents depth 3 levels: A.B.C
\setcounter{tocdepth}{3}

\begin{document}

\title{Sommarstugekoll \\
	EDA234, Group 2}
\author{Names \\
   Chalmers Kindergarten of Technology \\
   \texttt{email addresses}}

\maketitle

\begin{abstract}
\end{abstract}

\pagebreak

\tableofcontents

\section{Introduction}

\section{System Specification}

\section{System Description}

\section{Block Diagram}

\section{Block, Functionality}

	\subsection{Data Path}

	\subsection{Control Unit}

	\subsection{Functional Modules}

		\subsubsection{DTMF Module and MT8880}
	
		\subsubsection{Sound Module and ISD2560P}

		\subsubsection{Temperature Module and DS18S20}
			The temperature module of the CPLD is tasked with handling the serial communication with
			the DS18S20 Temperature sensors via the 1-Wire buses, and outputting the read temperatures
			to the sound module. 

			A read cycle consists of three stages, Initialization, Command issuing and Reading of data.
			Initialization is performed by pulling the bus low for 512 µs, then releasing the bus. The
			temperature sensor responds with a presence pulse by pulling the bus low for 106 µs, confirming
			its presence on the bus and its operational status. 

			Detecting the presence pulse, the master starts to issue a ROM Command (Skip ROM, 0xCC), followed
			by a short recovery time and then a Function Command (Convert Temperature, 0x44).
			The master then resets the temperature sensor as above, then sends another pair of ROM-Function 
			command, 0xCC and then 0xBE (Read Scratchpad).

			Ideally, the DS18S20 should now be ready to transmit the data stored on its on-chip memory of
			9 bytes.

		\subsubsection{Control Functions}

		\subsubsection{Interface}

\section{State Machines}
		\subsubsection{DS18S20 Serial Communication}
		\subsubsection{MT8880}
		\subsubsection{ISD2560P}

\section{Timing Diagrams}

\section{Fault Analysis}

\section{Appendix}

	\subsection{List of Components}

	\subsection{Circuit diagram}

	\subsection{PCB Layout}

	\subsection{Program listings}

	\subsection{List of Signals}

\end{document}
